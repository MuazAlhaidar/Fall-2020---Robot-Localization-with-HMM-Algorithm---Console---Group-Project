% Created 2020-11-11 Wed 15:49
% Intended LaTeX compiler: pdflatex
\documentclass[11pt]{article}
\usepackage[utf8]{inputenc}
\usepackage[T1]{fontenc}
\usepackage{graphicx}
\usepackage{grffile}
\usepackage{longtable}
\usepackage{wrapfig}
\usepackage{rotating}
\usepackage[normalem]{ulem}
\usepackage{amsmath}
\usepackage{textcomp}
\usepackage{amssymb}
\usepackage{capt-of}
\usepackage{hyperref}
\author{Muaz Alhaider and Zakariya Ahmed}
\date{\today}
\title{Program 2}
\hypersetup{
 pdfauthor={Muaz Alhaider and Zakariya Ahmed},
 pdftitle={Program 2},
 pdfkeywords={},
 pdfsubject={},
 pdfcreator={Emacs 27.1 (Org mode 9.3)}, 
 pdflang={English}}
\begin{document}

\maketitle
\tableofcontents


\section{Modules}
\label{sec:org368f7fe}

\subsection{Transitional Probabilty}
\label{sec:org375c0f6}
\begin{center}
$$Transitional\ Probabilty(Pos_i, Dir) = (Pos_{from\ left}:Drift(Left), Pos_{from\ straight}:Drift(Straight), Pos_{from\ right}:Drift(Right)): (\text{For smoothing})$$

$$Transitional\ Probabilty(Pos_i, Dir) = (Pos_{from\ left}:Drift(Left), Pos_{from\ straight}:Drift(Straight), Pos_{from\ right}:Drift(Right), Pos_{from\ behind}:Drift(Straight)): (\text{For prediction})$$


\emph{Left}, \emph{Right}, and \emph{Straight} are all positions defiend in relation to what \emph{Dir} is: if \emph{Dir} is EAST, then \emph{Left} is SOUTH, \emph{Right} is NORTH, and Straight is EAST.
\end{center}

Here, transitional probabilty has two forms: one which is all paths that converge to a point, and another where they diverge from a point. Prediction (as the name implies) wants all the possiple paths to a point, which is why we include \(Pos_{from\ behind}\). Smoothing however, does not require that, which is why it's not include.

\begin{verbatim}
  @enum SquareType OPEN CLOSED
@enum Direction WEST NORTH EAST SOUTH
@enum DriftType STRAIGHT LEFT RIGHT
const Drift=Dict(STRAIGHT=>.7, LEFT=>.15, RIGHT=>.15)
const AllDirects =(WEST, NORTH, EAST, SOUTH)

   # Gets the transitional probabilty of posistions going into a point,
# or transitional probablites from the point
# the first is used for prediction, the other for smoothing
function transprob( grid::Array{Array{Float64,1},1}, pos::Tuple{Int64, Int64}, dir::Direction, getforward=false) # => array of  (pos::Tuple{Int64, Int64}, Drift[Direction] * P(s), Drift[Direction])
	arr = []

	function _gen_parts(straightDir::Direction , leftDir::Direction, rightDir::Direction, behindDir::Direction, grid::Array{Array{Float64,1},1}, pos::Tuple{Int64,Int64} )
		parent_pos = []
		behind = move(pos, behindDir)
		straight = move(pos, straightDir)
		left = move(pos, leftDir)
		right = move(pos, rightDir)


		if( !notblocked(grid, straight)) # Bounce
			push!(parent_pos, (pos,Drift[STRAIGHT]*grid[pos[1]][pos[2]], Drift[STRAIGHT] ))
		elseif (getforward) # If this is smoothing, want the probabilty in front
			push!(parent_pos, (straight,Drift[STRAIGHT]*grid[straight[1]][straight[2]], Drift[STRAIGHT] ))
		end
		if( notblocked(grid, behind) && !getforward) # Prob of square behind current ot move to current
			push!(parent_pos, (behind,Drift[STRAIGHT]*grid[behind[1]][behind[2]], Drift[STRAIGHT]))
		end
		if(notblocked(grid, left)) # Get probabitly of the left pos coming to curretn
			push!(parent_pos, (left,Drift[LEFT]*grid[left[1]][left[2]], Drift[LEFT] ))
		else #Boucne from left
			push!(parent_pos, (pos,Drift[LEFT]*grid[pos[1]][pos[2]], Drift[LEFT] ))
		end
		if(notblocked(grid, right)) #Get probabilty of right pos coming to current
			push!(parent_pos, (right,Drift[RIGHT]*grid[right[1]][right[2]],Drift[RIGHT]))
		else #Bounce
			push!(parent_pos, (pos,Drift[RIGHT]*grid[pos[1]][pos[2]], Drift[RIGHT]))
		end
	end

	if(dir==WEST)
		arr=_gen_parts(WEST, SOUTH, NORTH, EAST, grid, pos)
	end
	if(dir==NORTH)
		arr=_gen_parts(NORTH, EAST, WEST, SOUTH, grid, pos)
	end
	if(dir==SOUTH)
		arr=_gen_parts(SOUTH, EAST, WEST, NORTH, grid, pos)
	end
	if(dir==EAST)
		arr=_gen_parts(EAST, SOUTH, NORTH, WEST, grid, pos)
	end
	arr
end

\end{verbatim}

\subsection{Prediction}
\label{sec:org07a2d75}
$$Prediction(Grid, Direction) = \left\{pos_i \in Grid | \sum^{Transiton\ Probability(pos_i, direction)}_{(Pos_j, DriftProb)} DriftProb \cdot P(Pos_j) \right \}$$

\(Prediction(Grid, Direction)\) (as the name implies) attempts to predict where the agent will be given previous infroamtion. It does this by transforming the grid by the equation  \(\sum^{Transiton\ Probability(pos_i, direction)}_{(Pos_j, DriftProb)} DriftProb \cdot P(Pos_j)\).This gets the probabilty of an agent drifting (or if direction is straight, accurantly going to) a point, and what is the probabtily the agent would be at the point \(Pos_j\).

\begin{verbatim}
    # Prediction is the sum of possiple transitional probabilties
# that can reach pos_i, * P(pos).j
function predict(grid::Array{Array{Float64,1}}, dir::Direction)
	tmp_grid = deepcopy(grid)
	for row in 1:6
		for col in 1:5
			if(notblocked(grid, (row,col)))
				val=sum([x[2] for x in transprob(grid, (row, col), dir)])
				tmp_grid[row][col]=val
			end
		end
	end
	tmp_grid

end

\end{verbatim}

\subsection{Evidence Contional Probabilty}
\label{sec:org447ad11}
\begin{center}
\(Evidence\ Contional\ Probability(Pos_i, Evidence) =  \prod^{\text{Directions} }_{dir=W} Sense(evidence[pos_i dir], actual[pos_i+dir])\)
\end{center}

This is the equation we use to get the evidence contional probability: it's the product of each the evidecne's value at a direction times what's actaully in the value of the direction. So if \emph{Left} has open, but evidecne says it's closed, it's 0.2. Taking the prodcut of all direction's sensed value and actual value, it will result in the Evidecne Contional Probabilty at \(Pos_i\) given \(Evidence\)

\begin{verbatim}
  # Given posistion and evidecne return probabilty of being in that posistion
const Sense=Dict(OPEN=>Dict(OPEN=>.8, CLOSED=>.2), CLOSED=>Dict(OPEN=>.25, CLOSED=>.75))

function evidence_Probabilty(grid::Array{Array{Float64,1},1}, pos::Tuple{Int64, Int64}, evidence::Tuple{SquareType, SquareType, SquareType, SquareType})
	prod = 1
	for i in 1:4
		tmp_pos = move(pos,AllDirects[i])
		block = notblocked(grid, tmp_pos)
		if (block)
			prod*= Sense[OPEN][evidence[i]]
		else
			prod*= Sense[CLOSED][evidence[i]]
		end
	end
	prod 
end
\end{verbatim}
\subsection{Filtering}
\label{sec:org750ac2c}
\begin{center}


\(pos_{s+1,i} = \frac{Evidecne\ Conditional\ Probabtily(pos_{s,i}, evidecne) \cdot P(pos_{s,i}) }{\sum^{all\ posistions}_{pos} Evidecne\ Conditional\ Probabtily(pos_{s,i}, evidecne) \cdot P(pos_{s,i})}\)

\(Filtering(Grid, Evidence) = \left\{pos_i \in Grid | \frac{P(pos_i) \cdot Evidence\ Conditional\ Probability(pos_i, Evidecne) }{\sum^{all\ posistions}_{pos} P(pos_i) \cdot Evidecne\ Conditional\ Probabtily(pos_i, evidecne) } \right\}\)
\end{center}

Filtering is a transformation upon the grid: each value gets transformed by the expression \(\frac{P(pos_i) \cdot Evidence\ Conditional\ Probability(pos_i, Evidecne) }{\sum^{all\ posistions}_{pos} P(pos_i) \cdot Evidecne\ Conditional\ Probabtily(pos_i, evidecne) }\), whcih for purposes of making it easier to talk about, will be expressed as \(Filter\ Step(pos_i, Evidence)\). \(Filter\ Step\) is condtional probabilty of each point times what the point was previously, and then dividng it  by the sum of all points on the grid. This operatoin is \(O(n)\), although more accuratly it's \(O(2n)\) because there's a minimal of iterating through each value twice.

\begin{verbatim}
   # Get the evidecne contional probabilty of each posistoin*Pos(s_i)
# Then divide each posistion with the evidnece conditonal probabilty
function filter(grid::Array{Array{Float64,1},1}, evidence::Tuple{SquareType, SquareType, SquareType, SquareType})
	tmp_grid = deepcopy(grid)
	for row in 1:6
		for col in 1:5
			if(notblocked(grid, (row,col)))
				tmp_grid[row][col]*=evidence_Probabilty(grid, (row, col), evidence)
			end
		end
	end
	total_sum = sum(sum(tmp_grid))
	# println("SUM: ", total_sum)
	tmp_grid / total_sum
end
\end{verbatim}
\subsection{Smoothing}
\label{sec:org966655b}


 \begin{math}
SmoothingPart(Grid,Previous,Direction,Evidence)= \\ \\
\{pos_{i} \in Grid|P(Pos_{i} )\cdot \sum ^{Transiton\ Probability(pos_{i} ,direction)}_{(Pos_{j} ,DriftProb)}\\ \\
P(Pos_{j} )\cdot DriftProb\cdot Evidence\ Conditional\ Probabtily(Grid,Pos_{j} ,Evidence)\}\\ \\
\end{math}



    \begin{math}
    Smoothing(Grid, Previous, Direction,Evidence) = \\
  \left\{pos_i \in Grid | \frac{SmoothPart(Grid, Previous,Direction, Evidence)}{\sum^{\text{All Positons}}_{pos} Smoothpart(Grid, Prevoius, Direction,Evidence)}  \right \}
\end{math}



Smoothing invovles

\begin{verbatim}
   # Get the transitional probabilty of a point going OUT, not in
# an it's conditional probabilty, with it's inital probabilty
# returns 2 things: B at pos, and B*p(s)
function smoothpart( last_grid::Array{Array{Float64,1}}, Bgrid::Array{Array{Float64,1}},  evidence::Tuple{SquareType, SquareType, SquareType, SquareType},  dir::Direction, pos::Tuple{Int64, Int64})
	parent_pos=transprob(grid, pos, dir, true )
	x=0
	# for i in parent_pos
	for (tmp_pos, prob, drift) in parent_pos
		# tmp_pos = i[1]
		# prob = i[2]
		# drift = i[3]
		y=evidence_Probabilty(grid, tmp_pos, evidence)* Bgrid[tmp_pos[1]][tmp_pos[2]]* drift
		x+=y
	end

	(x,   x *last_grid[pos[1]][pos[2]])
end

# Get the smoothing part for each posistion in grid
# Then divide the whoel grid by the sum of it's parts
function smooth( grid::Array{Array{Float64,1}}, last_grid::Array{Array{Float64,1}}, Bgrid::Array{Array{Float64,1}},  evidence::Tuple{SquareType, SquareType, SquareType, SquareType},  dir::Direction)
	SP = deepcopy(grid)
	B = deepcopy(Bgrid)
	for row in 1:6
		for col in 1:5
			if(notblocked(grid, (row,col)))
				val=smoothpart(last_grid, Bgrid, evidence, dir, (row,col))
				B[row][col] = val[1]
				SP[row][col] = val[2]
			else
				B[row][col]  = 0
				SP[row][col] = 0
			end
		end
	end
	# println("SUM: ", sum(sum(SP)))
	SP/=sum(sum(SP))
	# print_grid(SP); println(); print_grid(B); println()
	(SP, B)
end

\end{verbatim}
\section{Results}
\label{sec:org736d6d5}
The code outputs the following:
\begin{verbatim}
julia SUBMIT.jl
\end{verbatim}

\section{Screenshots}
\label{sec:org901cd2c}
\begin{center}
\includegraphics[width=.9\linewidth]{data/a9/3abd3f-f652-4b14-a3ef-d46d087ebe0c/screenshot-20201111-135323.png}
\end{center}
\begin{center}
\includegraphics[width=.9\linewidth]{data/a9/3abd3f-f652-4b14-a3ef-d46d087ebe0c/screenshot-20201111-135329.png}
\end{center}
\end{document}
